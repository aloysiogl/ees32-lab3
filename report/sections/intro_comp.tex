\section{Comparação Justa}
Nas práticas anteriores, para comparar a eficácia dos códigos utilizados, montavam-se gráficos da probabilidade de erro (eixo y) pela probabilidade de erro de bit no canal (eixo x) em escala logarítmica. Embora essa comparação forneça uma noção  a respeito da melhora obtida por um determinado código, a comparação não leva em conta que a introdução de redundância aumenta a energia que é utilizada para transmitir um bit de informação.

Poderíamos apenas aumentar a energia de uma codificação considerada menos eficiente anteriormente e ela teria probabilidade de erro menor. Dessa forma, a comparação justa deveria supor mesmo $\frac{E_i}{N_0}$. Supondo uma modulação BPSK, $p\left(\frac{E_b}{N_0}\right) = Q\left(\sqrt{\frac{2E_b}{N_0}}\right)$. Dessa forma, considerando mesmo $\frac{E_i}{N_0}$, tem-se $p\left(\frac{E_i}{N_0}\right) = Q\left(\sqrt{\frac{2E_i}{N_0}\frac{k}{n}}\right)$.

Os valores de $p$ utilizados nos experimentos anteriores foram remapeados supondo $k = n$. Dessa forma, obtiveram-se valores de $\frac{E_i}{N_0}$ para serem utilizados para cada tipo de codificação. Ressalta-se que tais valores foram muito parecidos com uma sequência de 0 a 10 igualmente espaçada. Os valores são apresentados abaixo.

0.1, 0.35416315, 0.82118721, 1.35277173, 2.10894229, 2.70594722,
	3.3174483, 4.1419075, 4.77476785, 5.41378309, 6.26609665, 6.91554181,
	7.56835261, 8.43569456, 9.09464674, 9.75571048, 10.63242365.
	
Antes de realizar a codificação essa sequência de valores passa pela função \textit{p\_map(k,n)}, a qual gera a sequência de probabilidades de erro correspondentes a um código de taxa dada por $k$ e $n$. Essa função realiza a operação $p\left(\frac{E_i}{N_0}\right) = Q\left(\sqrt{\frac{E_i}{N_0}\frac{k}{n}}\right)$ em cada um dos valores do vetor de $\frac{E_i}{N_0}$ acima. Desse modo, pode-se reutilizar exatamente o mesmo código das práticas anteriores apenas com um remapeamento dos valores de $p$.

Todos os gráficos gerados nesse relatório utilizam o método de comparação aqui descrito.