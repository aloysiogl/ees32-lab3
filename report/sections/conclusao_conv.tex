\section{Conclusão}

A maior dificuldade da implementação do codificador foi conseguir gerar o diagrama de transição de estados para efetuar o processamento da maquina, uma vez que realizar a convolução de sinais discretos possui um custo computacional mais elevado que cresce de acordo com o tamanho da mensagem.

Para o decodificador o maior desafio foi a implementação do algoritmo de Viterbi, uma vez que na primeira versão do código um pequeno detalhe de implementação gerou resultados errados retardando o desenvolvimento do decodificador. Outra dificuldade foi, no momento da implementação das variações, a implementação da segunda variação uma vez que foi necessário alterar o canal e alguns trechos do código foram alterados para operar com variáveis de ponto flutuante.