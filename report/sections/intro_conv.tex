\section{Código convolucionais}

Os códigos convolucionais baseiam-se na convolução de um sinal de entrada com um sistema codificador. Um codificador convolucional pode ser interpretado como uma máquina de estados que processa $k$ bits de entrada gerando $n>k$ bits de saída de forma que cada conjunto de bits gerados dependam do bit de processado e do estado atual da máquina.

O processo de codificação consiste em realizar o processamento dos bits pela máquina de estado, a cada iteração é alterado o estado da máquina e os bits de saída são processados de acordo com os polinômios geradores descrito pela máquina. No processo de decodificação é utilizado o algoritmo de Viterbi, que consiste em encontrar a sequencia de entrada da máquina que melhor aproxima a sequencia  de decodificação.